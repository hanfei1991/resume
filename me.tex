%-------------------------
% Resume in Latex
% Author : Sourabh Bajaj
% License : MIT
%------------------------

\documentclass[letterpaper,11pt]{article}

\usepackage{latexsym}
\usepackage[empty]{fullpage}
\usepackage{titlesec}
\usepackage{marvosym}
\usepackage[usenames,dvipsnames]{color}
\usepackage{verbatim}
\usepackage{enumitem}
\usepackage[hidelinks]{hyperref}
\usepackage{fancyhdr}
\usepackage[english]{babel}
\usepackage{tabularx}
\input{glyphtounicode}

\pagestyle{fancy}
\fancyhf{} % clear all header and footer fields
\fancyfoot{}
\renewcommand{\headrulewidth}{0pt}
\renewcommand{\footrulewidth}{0pt}

% Adjust margins
\addtolength{\oddsidemargin}{-0.5in}
\addtolength{\evensidemargin}{-0.5in}
\addtolength{\textwidth}{1in}
\addtolength{\topmargin}{-.5in}
\addtolength{\textheight}{1.0in}

\urlstyle{same}

\raggedbottom
\raggedright
\setlength{\tabcolsep}{0in}

% Sections formatting
\titleformat{\section}{
  \vspace{-4pt}\scshape\raggedright\large
}{}{0em}{}[\color{black}\titlerule \vspace{-5pt}]

% Ensure that generate pdf is machine readable/ATS parsable
\pdfgentounicode=1

%-------------------------
% Custom commands
\newcommand{\resumeItem}[2]{
  \item\small{
    \textbf{#1}{: #2 }
  }
}

% Just in case someone needs a heading that does not need to be in a list
\newcommand{\resumeHeading}[4]{
    \begin{tabular*}{0.99\textwidth}[t]{l@{\extracolsep{\fill}}r}
      \textbf{#1} & #2 \\
      \textit{\small#3} & \textit{\small #4} \\
    \end{tabular*}\vspace{-5pt}
}

\newcommand{\resumeSubheading}[4]{
  \vspace{-1pt}\item
    \begin{tabular*}{0.97\textwidth}[t]{l@{\extracolsep{\fill}}r}
      \textbf{#1} & #2 \\
      \textit{\small#3} & \textit{\small #4} \\
    \end{tabular*}\vspace{-5pt}
}

\newcommand{\resumeSubSubheading}[2]{
    \begin{tabular*}{0.97\textwidth}{l@{\extracolsep{\fill}}r}
      \textit{\small#1} & \textit{\small #2} \\
    \end{tabular*}\vspace{-5pt}
}

\newcommand{\resumeSubItem}[2]{\resumeItem{#1}{#2}\vspace{-4pt}}

\newcommand{\hhref}[2]{\href{#1}{\underline{#2}}}

\renewcommand{\labelitemii}{$\circ$}

\newcommand{\resumeSubHeadingListStart}{\begin{itemize}[leftmargin=*]}
\newcommand{\resumeSubHeadingListEnd}{\end{itemize}}
\newcommand{\resumeItemListStart}{\begin{itemize}}
\newcommand{\resumeItemListEnd}{\end{itemize}\vspace{-2pt}}

%-------------------------------------------
%%%%%%  CV STARTS HERE  %%%%%%%%%%%%%%%%%%%%%%%%%%%%


\begin{document}

%----------HEADING-----------------
\begin{tabular*}{\textwidth}{l@{\extracolsep{\fill}}r}
  \textbf{Han Fei} & Email : \hhref{mailto:hanfei19910905@gmail.com}{hanfei19910905@gmail.com}\\
  Github : \hhref{https://github.com/hanfei1991}{hanfei1991} & Mobile : +8618515307397 \\
\end{tabular*}

%-----------Biography------------
\section{About}
  I'm a senior software engineer working in the database field and an open source enthusiast. Since I joined PingCAP in 2016,
  I have endeavored to build the next generation database -- TiDB (\hhref{https://github.com/pingcap/tidb}{github/tidb}) project.
  I played a crucial role in developing several important modules of TiDB, including Optimizer, Computation Engine, TiFlash and TiFlow project.
  I'm adept at turning novel academic papers into productive code, and have rich experience in distributed databases.
  My technique stack mainly concentrates on golang and cpp but I'd love to ramp up new languages and technologies.

%-----------EXPERIENCE-----------------
\section{Experience}
  \resumeSubHeadingListStart

    \resumeSubheading
      {PingCAP}{Shanghai, China}
      {Senior Software Engineer}{April 2016 - Present}
      \resumeItemListStart
        \resumeItem{TiDB}
          {TiDB is an open source database inspired by Google Spanner and implemented by Go language. It has been widely used by enterprises and users from all around the world (\hhref{https://en.pingcap.com/customers/}{check out our customers}).
          I joined the TiDB team in a very early phase and led the development of the TiDB optimizer and SQL computation engine from 2016 to 2018. 
          Right now I'm the sixth top contributor in the TiDB community.}
      \resumeItemListStart
        \resumeItem{TiDB Optimizer}
          {TiDB Optimizer is a classic two-phase optimizer, which consists of rule-based and cost-based optimizing phases. I singlehandedly finished the early design and development. I also built the self-maintained statistics module and implemented the plan selection algorithm for TiDB optimizer (\hhref{https://drive.google.com/file/d/1HRK8XbsUMwCUaYR0Tof6HbLxzv-JohVA/view?usp=sharing}{slide}).
          I am an expert in the SQL optimizer field and have read lots of classic papers to keep refining TiDB optimizer.
          As a result, I specialize in integrating academic findings to implement complex algorithms.
          }
        \resumeItem{TiDB SQL Engine}
          {I developed TiDB's first SQL computation engine according to volcano model, including Join, Aggregation, Filter operator and expression evaluation framework.}
      \resumeItemListEnd
        \resumeItem{TiFlash}
          {TiFlash (\hhref{https://github.com/pingcap/tiflash}{pingcap/tiflash}) is the vectorized OLAP engine and columnar storage replica for TiDB. This project was originally based on ClickHouse database but developed their own storage engine and replication framework. 
          TiFlash extends TiDB's analysis capability and makes TiDB a real HTAP database (\hhref{https://www.vldb.org/pvldb/vol13/p3072-huang.pdf}{VLDB paper}). 
          Since 2018, I have been instrumental in the design and early development of TiFlash project. A lot of works are involved with compatibility with TiDB, including the DDL replication (\hhref{https://github.com/pingcap/tiflash/pull/103}{PR}) from TiDB, the MySQL Decimal type support (\hhref{https://github.com/pingcap/tiflash/pull/74}{PR ported from ClickHouse}), MySQL DateTime Type and MPP distributed computation framework (\hhref{https://github.com/pingcap/tiflash/pull/1111}{PR}).
          }
      \resumeItemListStart
        \resumeItem{MPP Framework}{
          As the main designer and main developer, I was responsible for TiFlash supporting massively parallel processing (\hhref{https://www.sisense.com/glossary/mpp-database/}{MPP}) architecture since 2020, which greatly expands TiFlash's computation capability (\hhref{https://drive.google.com/file/d/1OAoBk4_nw8mkMebRpEcu4y9w5Hh16Lfu/view?usp=sharing}{refer to the English design doc authored by me}). 
          }
      \resumeItemListEnd
       \resumeItem{TiKV Client} {TiKV Client (\hhref{https://github.com/tikv/client-c}{tikv/client-c}) is implemented by cpp and grpc. It is mainly used in TiFlash to read, write meta data from TiKV and to resolve locks for TiFlash query. I was in charge of TiKV Client's design, test, development and further maintenance.}
        \resumeItem{TiFlow}{TiFlow {\hhref{https://github.com/pingcap/tiflow}{(pingcap/tiflow)}} comprises CDC (\hhref{https://docs.pingcap.com/tidb/stable/ticdc-overview}{doc}) module which replicates data from TiKV to outside world and DM module (\hhref{}{doc}) which loads data from MySQL binlogs to TiDB. I joined TiFlow team in Sep 2021. }
        \resumeItemListStart
        \resumeItem{TiFlow Cloud Runtime} {
          My teammates and I designed a new TiFlow runtime on public cloud and developed a prototype on my private repo (\hhref{https://github.com/hanfei1991/microcosm}{microcosm}).
          This project is a scheduler and runtime to orchestrate CDC and DM tasks, and provides a unified interface for cloud users. I took part in the design, coding, test and deployment tools for an on-premise environment.
        }
        \resumeItemListEnd
      \resumeItemListEnd

    \resumeSubheading
      {Alibaba}{Beijing, China}
      {Software Engineer}{July 2014 - April 2016}
      \resumeItemListStart
        \resumeItem{Max-Compute Optimizer}
          {Max Compute (\hhref{https://www.alibabacloud.com/product/maxcompute}{link}) is the cloud data warehouse on Alibaba Cloud. I developed Max Compute rule-based optimizer.}
        \resumeItem{Stream Processing}{I built a distributed stream processing system in Alibaba.}
      \resumeItemListEnd

  \resumeSubHeadingListEnd


%-----------EDUCATION-----------------
\section{Education}
  \resumeSubHeadingListStart
    \resumeSubheading
      {Harbin Engineering University}{Harbin, China}
      {Bachelor of Engineering in Computer Science and Technologies}{2009 -- 2014}
  \resumeSubHeadingListEnd

%-------------------------------------------
\end{document}
